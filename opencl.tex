\section{OpenCL}

\subsection{Einleitung}

\begin{itemize}
\item CFD ist sehr anspruchsvoll was Rechenaufwand angeht
\item Traditionellerweise hat man (für Endanwender) auf CPUs programmiert
\item CPUs haben seit kurzem immerhin mehrere Kerne, sodass OpenMP und MPI Sinn ergeben
\item Die Entwicklung zu mehreren Kernen wurde von der Stromsparüberlegung angetrieben (viele Kerne mit niedrigeren Frequenzen leisten dasselbe, sind aber energieeffizienter)
\item Dadurch außerdem kleinere, spezialisiertere Prozessoren (weniger "wasted transistors")
\item Aber GPUs haben theoretisch wesentlich mehr Rechenpower\cite{Guide2012}
\item Grade CFD ist sauschnell
\item Aber natürlich brauch man Probleme, die gut parallelisierbar sind
\item Definition Concurrency: "A software system is concurrent when it consists of more than one stream of operations that are active and can make progress at one time."
\item Definition Parallelism: "When concurrent software runs on a computer with multiple processing elements so that threads actually run simultaneously, we have parallel computation. Concurrency enabled by hardware is parallelism."
\item In diesem Abschnitt Architektur von OpenCL deutlich machen und klären welche Probleme damit gut lösbar sind.
\end{itemize}

\subsection{Architektur}

\begin{itemize}
\item Dezember 2008 released
\item Momentan bei Version 1.2, obwohl das noch wenige implementieren
\item Parallelisierbarkeit erklären: data parallelism und task parallelism
\item Dataparallelism: Idee ist, dass man eine Sammlung von Daten hat, die
nebenläufig upgedatet werden können. Parallelität wird nun grade dadurch
erzeugt, dass man denselben Instructionstream auf jedes Datenelement abfeuert
(nicht etwa für jedes Datenelement 'nen eigenen Thread hat, der sich beliebig
anders verhält als der nächste Thread). Beispiel: Addition zweier Vektoren
(SIMD allgemein).
\item Taskparallelism: Eignet sich vielleicht eher für Traversierung von Graphen
oder so.
\item Platform-Model erklären
    \begin{itemize}
            \item "High level description of the heterogenous system"
            \item Host (gibt nur einen)
            \item Host enthält Devices
            \item Devices enthalten Compute Units
            \item Compute Units enthalten Processing Elements
            \item Host program, was auf Host läuft
            \item Mehrere Kernel laufen auf jeweils einem Device
            \item Execution Model klärt, wie Kernel ausgeführt werden
    \end{itemize}
\item Execution-Model
    \begin{itemize}
            \item "abstract representation of how streams of instructions execute on the heterogenous platform"
            \item Kernelexecution löst die Erstellung eines ganzzahligen "Gitters" aus, jeder Kernel kriegt eine Gitterzelle, die globale ID und wird zum "Work-Item"
            \item Workitems werden in Workgroups zusammengefasst (haben auch ID, bilden eine gröbere Struktur, haben geteilten Speicher, werden auf derselben Computeunit ausgeführt)
            \item Command-Queues, schedulen Kernel, Memorycommands, Synchronisierungskommandos, in-order, out-of-order
            \item Barrieren, Warps
    \end{itemize}
\item Memory-Model
    \begin{itemize}
            \item "the collection of memory regions withing OpenCL and how they interact during an OpenCL computation"
            \item Bufferobjects (eindimensional, beliebige Datentypen), Imageobjects, mehrdimensional usw., Inhalt "versteckt"
            \item Host memory
            \item Global, local, private, constant
            \item Zusammenspiel mit OpenGL
    \end{itemize}
\item Programming model
\begin{itemize}
    \item High level abstractions for algorithm programming
    \item OpenCL-C erläutern
\end{itemize}
\end{itemize}

