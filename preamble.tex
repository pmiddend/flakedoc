\section{Danksagungen}

Hiermit möchte ich allen Personen danken, die mich bei der Erstellung der Arbeit
unterstützt haben:

\begin{itemize}
\item Herrn Prof. Dr. Oliver Vornberger für die Tätigkeit als Erstgutachter und
für die Bereitstellung der interessanten Thematik.
\item Frau Prof. Dr. Sigrid Knust, die sich als Zweitgutachterin zur Verfügung gestellt hat.
\item Frau Jana Lehnfeld und Herrn Henning Wenke für das wertvolle Feedback
während der Verfassung der Arbeit.
\end{itemize}

\section{Zusammenfassung}

Die vorliegende Arbeit entstand in der Arbeitsgruppe Medieninformatik an der Universität Osna
im Bereich Computergrafik und beschäftigt sich mit prozeduraler Modellierung. Kern
der Arbeit ist die Generierung einer Schneedecke, um Wetterverhältnisse anhand
von Wetterdaten naturgetreu darstellen zu können. Dazu wird der Anzatz einer
Repräsentation durch Voxel in einem regelmaßigen Voxelgitter verfolgt. Für die
realistische Verteilung des Schnees sorgt dabei die Simulation zufalligen
Schneefalls sowie ein Stabilitatstest, der die Entstehung unnaturlich hoher
Schneeturme verhindert. Abschließend wird das Voxelgitter unter Verwendung des
sogenannten Marching-Cubes-Algorithmus’ in eine schneeahnliche Oberflache
uberfuhrt.
