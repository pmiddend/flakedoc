\section{Einleitung}

Die Arbeit befasst sich mit der Modellierung und Entwicklung einer Methode zur
Simulation von Wind, sowie einem Partikelsystem für Schneeflocken. Zu Anfang
soll ein Überblick über die Zielsetzung gegeben werden. Danach folgt eine kurze
Zusammenfassung vorhandener Simulationsmethoden. Hier wird die Entscheidung für
eine der Methoden begründet vorgestellt. Schließlich folgt ein Überblick über
den Aufbau der Arbeit.

\subsection{Motivation}

...

\subsection{Zielsetzung}

...

\subsection{Aufbau}

Obacht, was folgt ist aus dem eigentlich geplanten Kapitel
``Implementierung'' entnommen, daher fehlt noch was am Grundaufbau.

In diesem Kapitel soll auf die Details der Implementierung eingegangen werden. Es
wird zunächst OpenGL eingeführt, was zur Visualisierung eingesetzt wird. Danach
wird OpenCL eingeführt, was für die eigentlichen Berechnungen zuständig ist.

Danach das Verfahren von Stam mittels OpenCL umgesetzt. Als Ausgabe erhält man
ein Schnee-Vektorfeld, was dann weiterverwendet werden kann. Hier wird auch auf
Hindernisse wie Gebäude eingegangen. Dazu gehört das Laden der Hindernisse aus
obj-Dateien sowie die Anzeige mittels OpenGL, sowie die Einspeisung in die
Simulation.

Anschließend wird der fallende Schnee modelliert. Hierzu soll zunächst die
Entstehung von Schnee in der Natur erläutert werden, sowie physikalische
Hindergründe bezüglich fallendem Schnee.

Diese idealen physikalischen Gegebenheiten werden dann in ein einfacheres
Modell transformiert, welches für das Partikelsystem verwendet wird. Zu diesem
Partikelsystem gehört neben der Simulation auch die Visualisierung, für die in
dieser Arbeit Pointsprites verwendet werden.

Im letzten Abschnitt wird schließlich auf die Modellierung der Schneedecke
eingegangen, wobei die Ergebnisse von Manuel Schwarz \cite{Schwarz2012} eine
tragende Rolle spielen. Außerdem werden noch einige weitere
Anwendungsmöglichkeiten der Fluidsimulation wie die Simulation von Rauch
angesprochen.
