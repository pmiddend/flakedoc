\section{Einleitung}

\subsection{Motivation}

Die Idee, Schneefall zu modellieren, entstand im Rahmen der
Masterprojektgruppe \PimiddyQuotes{virtueller Campus} an der
Universität Osnabrück im Wintersemester 2011/12. Die Projekgruppe
befasste sich mit dem Nachbau des Campus am Westerberg der Universität
Osnabrück unter Anwendung aktueller Methoden der Computergrafik.

Die Zielsetzung der Projektgruppe enthielt auch die Visualisierung von
natürlichen Phänomenen wie einem dynamischen Wechsel von Tag und
Nacht, sowie wechselnde Wetterverhältnisse. In diesem Zusammenhang
entstand die Idee, die Modellierung von Schneefall auszulagern und
separat von der Projektgruppe zu entwickeln. Daraus entstanden zwei
Arbeiten: die vorliegende und die Bachelorarbeit von Manuel Schwarz
(\cite{Schwarz2012}).

Die Arbeit von Schwarz befasst sich mit den verschiedenen Arten, eine
dynamisch wachsende und schrumpfende, dreidimensionale Schneedecke zu
modellieren. Als Resultat wurde ein Mesh-basiertes Verfahren
implementiert, das realistische Ergebnisse mit Stabilitätsbedingungen
für den Schnee liefert. Die Visualisierung der Schneedecke stand in
dieser Arbeit allerdings nicht im Vordergrund.

Das Ziel der vorliegenden Arbeit ist die Modellierung von fallendem
Schnee, also der Bewegung von Schneeflocken im Wind. In heutigen
interaktiven Anwendungen wird meist nur der optische Eindruck von
Schneeflocken und Schneefall angenähert. In dieser Arbeit soll im
Gegensatz dazu ein Ansatz vorgestellt werden, der darauf basiert, den
Wind an sich in in Echtzeit zu modellieren und die Flocken dann anhand
des Feldes fortzutragen.

Es stellte sich die Frage, ob dies überhaupt in Echtzeit möglich sei,
oder ob die Berechnungen eher verzögert im Hintergrund ablaufen
sollten. Allerdings stellte sich heraus, dass mit der Hilfe moderner
Grafikkarten (GPUs) durchaus Echtzeitframeraten zu erreichen sind. Das
Ziel bestand darin, die entsprechenden Verfahren umzusetzen und darauf
aufbauend ein Modell für die Bewegung von Schneeflocken zu entwickeln,
welches auf das berechnete Windfeld zurückgreift.

Im Laufe der Implementierung wurde als weiteres Ziel die Integration
der Ergebnisse von \cite{Schwarz2012} hinzugefügt, um die gefallenen
Schneeflocken in die Bildung der Schneedecke einfließen zu lassen.


\subsection{Aufbau}

Da sowohl die Berechnungen für die Bewegung der Schneeflocken als auch
die strömungsmechanischen Gleichungen gewisse mathematische
Grundkenntnisse voraussetzen, werden im \autoref{sec:mathematics}
einige Mathematische Definitionen und Verfahren eingeführt. Diese
Arbeit setzt aber ansonsten keine höheren mathematischen Vorkenntnisse
voraus und sollte auch ohne Vorkenntnisse aus der Strömungsmechanik
verständlich sein.

Aufgrund der hohen Komplexität und dem Umfang des Themas ist die
Behandlung der strömungsmechanischen Zusammenhänge in drei Abschnitte
aufgeteilt. \autoref{sec:navier_stokes} gibt zunächst eine Einführung
in die Navier-Stokes-Gleichungen. Diese Gleichungen sind sehr
allgemein gefasst und beschreiben jegliche Art von Strömungen. Um die
Erklärung nicht zu kompliziert zu gestalten wird eine einfachere Form
der Gleichungen vorgestellt. Zudem werden nur die Teile näher
beleuchtet, die für die Modellierung von Wind eine Rolle spielen. Der
Abschnitt soll vor allem ein intuitives Verständnis der einzelnen
Bestandteile geben, welches für das Verständnis der Algorithmen auch
ausreichend ist.

\autoref{sec:stam} stellt das Verfahren von Stam vor. Es wird zunächst
erklärt, inwiefern sich das Verfahren von bisherigen Verfahren
unterscheidet und was es auszeichnet. Danach folgt ein allgemeiner
Überblick über die Bestandteile. Die darauf folgenden Abschnitte
erklären diese Einzelteile, wobei aber noch wenig auf eine
Implementierung eingegangen wird. Am Ende des Abschnitts sollte die
Arbeitsweise des Verfahrens klar sein, sodass es theoretisch in einer
beliebigen Programmiersprache umsetzbar wäre.

\autoref{sec:opencl} erklärt die Funktionsweise von OpenCL, wobei die
Teile nur angeschnitten werden, die für die Implementierung der
Algorithmen keine Rolle spielen. Am Ende des Abschnittes findet sich
ein simples, aber vollständiges OpenCL-Programm, welches den Ablauf
und die Interaktion zwischen GPU und CPU verdeutlicht.

In \autoref{sec:implementation_wind} wird schließlich OpenCL
eingesetzt, um das Verfahren von Stam auf die Grafikkarte zu
übertragen. Es wird viel auf \autoref{sec:stam} verwiesen und stärker
auf die Schwierigkeiten in der Implementierung eingegangen als die
Erläuterung der prinzipiellen Funktionsweise. Auch einige
Performanceoptimierungen werden angesprochen.

\autoref{sec:implementation_snowflake} widmet sich dem fallenden
Schnee. Es zunächst auf die physikalischen Ursachen für die
Entsteheung von Schnee eingangen. Danach wird die Visualisierung mit
OpenGL angesprochen. Der Rest des Abschnitts befasst sich mit den
physikalischen Besonderheiten von Schneeflocken. Als Ergebnis wird ein
OpenCL-Programm erstellt, was ein Schnee-Partikelsystem umsetzt.

\autoref{sec:fallen_snow} behandelt die Modellierung von gefallenem Schnee.
